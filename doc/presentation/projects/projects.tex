\documentclass{article}
\usepackage{polski}
\usepackage[latin2]{inputenc}
\usepackage{fancyhdr}

\begin{document}

\lhead{\sc Nemerle -- Propozycje Podprojektów}
\chead{}
\rhead{\thepage}
\cfoot{}
\rfoot{}
\lfoot{}
\pagestyle{fancy}

\title{ {\sc Nemerle} \\ Propozycje Podprojektów}
\author{Michał Moskal, Paweł Olszta, Kamil Skalski}
\date{\today}

\maketitle

\section{Wstęp}

W ramach grantu udzielonego przez Microsoft Research na rozwój języka Nemerle
zostanie uruchomionych kilka podprojektów -- związanych z przedmiotem
dotacji.

Chcielibyśmy zaangażować do pracy nad Nemerle jak największą liczbę studentów 
naszego instytutu. Pracy rzecz jasna płatnej -- przewidywana wysokość wynagrodzenia 
za projekt wynosi od 200 do 500 euro. Dokładne kwoty (oraz opisy projektów) 
zostaną ustalone wkrótce.

Prosimy o kontakt zainteresowanych uzyskaniem szczegółowych informacji --
listowny (devel-pl@nemerle.org) lub osobisty.

Więcej informacji o Nemerle można znaleźć na stronie projektu: \verb,http://nemerle.org/,.


\section{Wtyczka Nemerle do Visual Studio}

\begin{itemize}
\item kolorowanie składni
\item integracja z kompilatorem
\item możliwość dodawania plików \verb,.n, do projektów
\item zawijanie kodu
\item IntelliSense, gdy kompilator będzie na to gotowy
\end{itemize}


\section{Wtyczka Nemerle do MonoDevelop}

\begin{itemize}
\item kolorowanie składni
\item integracja z kompilatorem
\item możliwość dodawania plików \verb,.n, do projektów
\item zawijanie kodu, o ile okaże się to możliwe
\item włączenie wtyczki w oficjalne źródła MonoDevelop
\item IntelliSense, gdy kompilator będzie na to gotowy
\end{itemize}


\section{Konwerter C\# do Nemerle}

\begin{itemize}
\item generowany kod ma być czytelny
\item komentarze mają być zachowane
\item inteligentny algorytm przekładu struktur kontrolnych
\item najlepiej testować na dużych programach :-)
\end{itemize}


\section{Interfejs Nemerle -> C\#}

Korzystanie z poziomu C\# z list, opcji etc. zadeklarowanych w Nemerle
jest dosyć nieeleganckie. Proponuję stworzyć w C\# zbiór klas umożliwiających
łatwe korzystanie z bibliotek napisanych w Nemerle.


\section{Konwerter Nemerle do C\#}

Nie chodzi tu o kompilator, tylko o wskazanie możliwości 'ucieczki'
z Nemerle do C\#, gdyby zaszła taka konieczność.

\begin{itemize}
\item generowany kod ma być czytelny
\item komentarze mają być zachowane
\item inteligentny algorytm przekładu struktur kontrolnych
\item najlepiej testować na dużych programach :-)
\end{itemize}


\section{CodeDom generator dla Nemerle}

Generator powinien obsługiwać wystarczająco dużo konstrukcji by dało się zintegrować 
Nemerle z ASP.NET, które korzysta z CodeDom. Może być to trudne (w Nemerle na
przykład nie ma instrukcji GOTO lub podobnej) -- należy przeanalizować, z których
konstrukcji korzysta ASP.NET (na przykładzie źródeł Mono lub Rotora) i jak ew.
poradzić sobie z brakiem istotnych konstrukcji.


\section{Sioux}

Sioux to serwer HTTP napisany w Nemerle. Przedmiotem projektu jest
stworzenie, w oparciu o Siouxa, serwera aplikacji Nemerle. Vide projekty
JBoss, Cocoon, Velocity.

\end{document}

% vim: language=polish
