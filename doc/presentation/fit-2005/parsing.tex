\documentclass[10pt]{beamer}
\usepackage{beamerthemeshadow}
%\usepackage{beamerthemesidebar}
\usepackage{graphicx}
\usepackage{polski}
\usepackage{verbatim}
\usepackage[utf8]{inputenc}

\newcommand{\net}[0]{{\tt .NET}}
\newcommand{\kw}[1]{{\textcolor{kwcolor}{\tt #1}}}
\newcommand{\ra}{\texttt{ -> }}

\definecolor{kwcolor}{rgb}{0.2,0.4,0.0}
\definecolor{lgray}{rgb}{0.8,0.8,0.8}

\title{Parsowanie dynamicznie rozszerzalnej składni}
\author{Kamil Skalski}
\institute{Forum Informatyki Teoretycznej \\ Karpacz}
\date{16 kwietnia 2005}


\begin{document}

\section{Zaczynamy!}

\frame{\titlepage}

\frame{
\frametitle{Rozszerzenia języków programowania}
\begin{itemize}
  \item języki powinny być tworzone z myślą o rozwoju
  \item dlaczego nie dać tej możliwości użytkownikom?
  \item nowa semantyka starych konstrukcji języka - makra
  \item nowa składnia, nowe konstrukcje - rozszerzenia składniowe
\end{itemize}
}

\frame{
\frametitle{Co to jest makro?}
  Funkcja o sygnaturze $f : \epsilon \rightarrow \epsilon$, gdzie 
  $\epsilon$ to drzewo składniowe w reprezentacji używanej przez kompilator.

  Makra są wykonywane przez kompilator w czasie analizy drzewa programu,
  umożliwiając zmianę, generację i analizę kodu w łatwy sposób.

  Makra są uruchamiane po napotkaniu wyrażenia o postaci 
  $nazwa_makra (parametry)$ lub składni dodanej przez załadowane
  makra.
}

\frame{
\frametitle{Przykład}
% \begin{verbatim}
%    macro repeat_times (count, body)
%    syntax ('repeat', '(', count, ')', body)
%    {
%      <[ for (mutable i = 0; i <= $count; i++) $body ]>
%    }
% \end{verbatim}
}

\section{To jest już koniec}
\frame{
\frametitle{Podsumowanie}
}


\end{document}

% vim: language=polish
