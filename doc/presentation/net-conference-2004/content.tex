\newcommand{\net}[0]{{\tt .NET}}
\newcommand{\kw}[1]{{\textcolor{kwcolor}{\tt #1}}}

\definecolor{kwcolor}{rgb}{0.2,0.4,0.0}
\definecolor{lgray}{rgb}{0.8,0.8,0.8}

\title{Nemerle}
\author{Micha{\l} Moskal, Kamil Skalski}
\institute{Instytut Informatyki Uniwersytetu Wroc�awskiego \\
Konferencja grup \net, Wroc�aw}
\date{2 grudnia 2004}


\begin{document}

\section{Wst�p}

\frame{\titlepage}

\frame{
\frametitle{Cechy Nemerle}
\begin{itemize}
  \item j�zyk programowania wysokiego poziomu
  \item od pocz�tku projektowany z my�l� o \net
  \item funkcjonalny i obiektowy
  \item pot�ny system metaprogramowania
  \item system asercji
\end{itemize}
}


\frame{
\frametitle{Czemu \net\ ?}

\begin{itemize}
  \item szeroki dost�p do bibliotek
  \item �rodowisko uruchomieniowe (od�miecanie, JIT)
  \item przeno�ne pliki wykonywalne (Microsoft \net, Mono, DotGNU, Rotor)
  \item dynamiczne �adowanie klas
  \item dynamiczna generacja kodu
\end{itemize}
}

\frame{
\frametitle{Po co to to?}

\begin{itemize}
  \item u�ywanie r�nych funkcjonalno�ci platformy \net\ jest znacznie �atwiejsze
        w C\# ni� w adaptacjach istniej�cych j�zyk�w funkcjonalnych
  \item �atwo�� definicji kontra �atwo�� u�ycia
  \item �atwy dost�p do cech imperatywnych
  \item prosty system obiektowy (bezpo�rednio z \net)
\end{itemize}
}

\frame{
\frametitle{Wi�c jak z tym j�zykiem?}

\begin{itemize}
  \item sk�adnia przypomina C\#, szczeg�lnie na poziomie definicji klas i metod
  \item wyra�enia syntaktycznie z C\#, semantycznie z ML-a
  \begin{itemize}
    \item brak instrukcji -- tylko wyra�enia
    \item dopasowanie wzorca na wariantach
    \item funkcje jako obywatele pierwszej kategorii
  \end{itemize}
\end{itemize}
}


\section{Wszyscy lubimy przyk�ady}

\frame[containsverbatim]{
\frametitle{Cze��}

\begin{verbatim}
class Hello {
  public static Main () : void {
    System.Console.Write ("Hello world!\n");
  }
}
\end{verbatim}
}


\frame[containsverbatim]{
\frametitle{Brak inferencji}
\begin{verbatim}
// C#
public static void SendMessage (byte[] addr, int port, 
                                string data)
{
  IPEndPoint ip = new IPEndPoint (new IPAddress (addr), port);
  TcpClient client = new TcpClient (ip);
  NetworkStream str = client.GetStream ();
  byte[] data = Encoding.UTF8.GetBytes (data);
  str.Write (data, 0, data.Length);
  client.Close ();
}
\end{verbatim}
}

\frame[containsverbatim]{
\frametitle{Inferencja}
\begin{verbatim}
// Nemerle
public static SendMessage (addr : array [byte],
                           port : int, data : string) : void
{
  def ip = IPEndPoint (IPAddress (addr), port);
  def client = TcpClient (ip);
  def str = client.GetStream ();
  def data = Encoding.UTF8.GetBytes (data);
  str.Write (data, 0, data.Length);
  client.Close ();
}
\end{verbatim}
}


\frame[containsverbatim]{
\frametitle{Assertions}
\begin{verbatim}
interface ICollection {
  Contains ([NotNull] key : object) : bool;
  
  [Require (!Contains (key))]
  Add ([NotNull] key : object, value : object) : bool;
		 
  Size : int { get; }

  [Ensure (Size == 0)]
  Clear () : void;
}
\end{verbatim}
}

\section{Macros}
\frame{
\frametitle{Macros}
\begin{itemize}
  \item dynamically loaded compiler modules
  \item compilation-time or run-time execution
  \item written in Nemerle
  \item work on syntax trees of expressions and types
  \item can read external files, query database etc.
\end{itemize}
}


\frame{
\frametitle{Uses of macros}
\begin{itemize}
  \item specialized sublanguages ({\tt printf}, {\tt scanf}, regular expressions,
    SQL, XML, XPath)
  \item generation of syntax trees out of external files and {\it vice versa} 
       (Yacc, Burg, types from DTD, documentation generation)
  \item generating trees from other trees (serialization, specialization of code)
  \item interpreter implementation
\end{itemize}
}

\section{Podsumowanie}
\frame{
\frametitle{Status}

\begin{itemize}
  \item kompilator kompiluje sam siebie
  \item wydana wersja 0.2
  \item CLS consumer/producer
  \item biblioteka standardowa
  \item makra
  \item \textcolor{blue}{\tt http://nemerle.org/}
\end{itemize}
}

\frame{
\frametitle{TODO}

\begin{itemize}
  \item u�y� .NET generics
  \item dalsze prace nad systemem makr (AOP)
  \item rozszerzenie dokumentacji
  \item definicje formalne (semantyka, system typ�w, inferencja typ�w)
\end{itemize}
}

\end{document}

% vim: language=polish
